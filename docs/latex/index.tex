\hypertarget{index_intro_sec}{}\section{Introduction}\label{index_intro_sec}
This page contains the documentation for our elevator project.\hypertarget{index_summary}{}\section{Implementation summary}\label{index_summary}
\hypertarget{index_fsm}{}\subsection{Finite State Machine (\+F\+S\+M)}\label{index_fsm}
Our elevator implementation is based on a finite state machine. All states in the F\+SM is in itself its own function and thus the current state is simply determined by which function is called. Each function returns a return code which is part of determining the next state function to call. The F\+SM includes a transition table which contains all possible state transitions and a lookup function. This lookup function uses the transition table to find the next state based on the current state and the return code given by the current state function. In the main loop inside the main function in \hyperlink{main_8c_source}{main.\+c} the state is then updated and the next state function is called. The F\+SM is implemented in the \hyperlink{fsm_8c_source}{fsm.\+c} file.\hypertarget{index_exec}{}\subsection{Execution handler}\label{index_exec}
This project also has an execution handler module which is responsible for the actions of the elevator. The thought here is that the state functions in the F\+SM mostly just calls function in the execution handler to handle orders, get return codes, etc. The execution handler is implemented in the \hyperlink{exec_8c_source}{exec.\+c} file.\hypertarget{index_order_and_scheduler}{}\subsection{Order handling system}\label{index_order_and_scheduler}
The order handling system is based around two queues, one for orders made inside the elevator and one for orders made outside the elevator. The reason for the seperation is because we believe that orders made inside the elevator should be executed before orders made outside the elevator, and thus having two seperate queues eases the logic for deciding which order to handle. To handle these queues our A\+PI includes a scheduler and an order handler.\hypertarget{index_scheduler}{}\subsubsection{Scheduler}\label{index_scheduler}
The scheduler handles the queue specific tasks of adding and deleting orders. It is based on a first in, first out principle to prioritize the orders. The scheduler also handles all the logic of validating the orders. This validation is specific for this elevator. The scheduler is implemented in the \hyperlink{scheduler_8c_source}{scheduler.\+c} file.\hypertarget{index_order}{}\subsubsection{Order handler}\label{index_order}
The order handler serves as a link between the execution handler and the scheduler and does most of the ordering tasks, specific to this elevator such as removing all orders at all floor. The reasoning for this is to get another level of abstraction in the queue system. The implementation of the order handler can be found in the \hyperlink{order_8c_source}{order.\+c} file. 